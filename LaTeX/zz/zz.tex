\documentclass[UTF8]{article}
\usepackage{ctex}
\usepackage{xcolor}
\begin{document}
    四个全面:\\
    1.全面建设社会主义现代化国家\\
    2.全面深化改革\\
    3.全面依法治国\\
    4.全面从严治党\\
    主题:推动高质量发展\\
    主线:深化供给侧结构性改革\\
    动力:改革创新\\
    根本目的:满足人民日益增长的美好生活需要\\
    (格局)主体:国内大循环\\
    (格局)基点:扩大内需\\
    创新:我国现代化建设全局中的核心(引领新常态根本之策,现代化经济体系战略支撑,引领发展的第一动力)\\
    科技自立自强作为国家发展的战略支撑,摆在各项规划任务的首位\\
    3个没有改变:\\
    1.发展具有多方面的优势和条件\\
    2.发展具有强劲韧性\\
    3.经济长期向好的基本面\\
    \\
    建议稿起草特点:坚持发扬民主\(\quad\)开门问策\(\quad\)集思广益\\
    5个原则:继承和创新\(\quad\)政府和市场\(\quad\)开放和自主\(\quad\)发展和安全\(\quad\)战略和战术\\
    \\
    <<建议>>是开启全面建设社会主义现代化国家新征程,向第二个
    百年奋斗目标进军的\textcolor{red}{纲领性文件,经济社会发展的行动指南}\\
    <<建议>>的核心要义\textcolor{red}{新发展阶段,新发展理念,新发展格局}\\
    实现2035远景目标要\textcolor{red}{改革,开放,创新}\\
    \textcolor{red}{改革是推动发展的强大动力}\\
    \textcolor{red}{开放是促进发展的必然选择}\\
    \textcolor{red}{创新是引领发展的第一动力}\\
    \\
    保护和激发市场主体活力.市场主体是经济的力量载体,他们是经济活动的主要参与者,
    就业机会的主要提供者,技术进步的主要推动者.保护市场主体就是保护社会生产力.\\
    \\
    新发展格局是根据我国\textcolor{red}{发展阶段,环境,条件变化}提出来的,是
    \textcolor{red}{重塑我国国际合作和竞争新优势战略抉择,是事关全局的系统性深层次变革}\\
    
\end{document}