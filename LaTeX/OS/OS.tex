\documentclass[UTF8]{article}
\usepackage{ctex}
\begin{document}
    为解决人机矛盾及CPU和I/O设备之间速度不匹配的矛盾\\
    脱机I/O:脱离主机的情况下进行\\
    联机I/O:在主机的直接控制下进行\\
    \\
    硬实时任务:系统必须满足任务对截止时间的要求\\
    软实时任务:也有一个截止时间,但不严格\\
    \\
    从交互性,及时性和可靠性,将分时系统和实时系统进行比较:\\
    交互性:实时信息处理系统虽然也具有交互性,但这里人与系统的交互仅限于访问系统中某些特定的专用服务程序。它不像分时系统那样能向终端用户提供数据处理和资源共享等服务。\\
    及时性:实时信息处理系统对实时性的要求与分时系统类似,都是以人所能接受的等待时间来确定的;而实时控制系统的及时性,则是以控 制对象所要求的开始截止时间或完成截止时间来确定的,一般为秒级到毫秒级,甚至有的要低于100微秒。\\
    可靠性:分时系统虽然也要求系统可靠,但相比之下,实时系统对可靠性的要求更高。因为任何差错都可能带来巨大的经济损失,甚至是无 法预料的灾难性后果,所以在实时系统中,往往都采取了多级容错措施来保障系统的安全性及数据的安全性。\\
    \\
    
\end{document}